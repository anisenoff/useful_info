\documentclass[letterpaper,twocolumn,10pt]{article}
% placeholder style
\usepackage{usenix-2020-09}

\usepackage{graphicx} % For figures
\DeclareGraphicsExtensions{.pdf,.eps,.png,.jpg} % For figures
\graphicspath{{figures/}} % paths to graphics folders
\usepackage{amsmath, amssymb} % Math symbols
\usepackage{booktabs} % For formal tables
\usepackage{multirow}
\usepackage{balance} % Balance references
\usepackage{xspace} % Spacing
\usepackage{hyperref} % Hyperlinks
\usepackage{cite} % Citations
\usepackage{array} % Pretty tables



\pagenumbering{gobble} % Removes page numbers




\newcommand{\todo}[1]{{\color{red}\bf [To-do: {#1}]}}


\newcommand{\nameone}[1]{{\color{orange}\bf [name1: {#1}]}}
\newcommand{\nametwo}[1]{{\color{cyan}\bf [name2: {#1}]}}


% PDF Metadata
\hypersetup{%
  pdfauthor={Anonymous Authors},
  pdftitle={Title},
  pdfkeywords={privacy; minors; advertising}
  pdfdisplaydoctitle=true, % For Accessibility
  bookmarksnumbered,
  pdfstartview={FitH},
  colorlinks,
  citecolor=black,
  filecolor=black,
  linkcolor=black,
  urlcolor=black,
  breaklinks=true,
  hypertexnames=false
}

\begin{document}
\month=1 \day=1 \year=2001 % 
%don't want date printed
\date{}

% make title bold and 14 pt font (Latex default is non-bold, 16 pt)
\title{\Large \bf Targeting Minors}




\author{
  {\rm Anonymous Authors}\\
%  {emails}\\
%  Carnegie Mellon University\\
}

\maketitle

% As a general rule, do not put math, special symbols or citations
% in the abstract
\begin{abstract}
\input{sections/00-abstract}
\end{abstract}



\section{Introduction}\label{sec:intro}

placeholder \cite{englehardt-2016-Online}
\section{Related Work}\label{sec:relwork}
\section{Methods}\label{sec:methods}
\section{Results}\label{sec:results}
\section{Discussion}\label{sec:discussion}

\section*{Acknowledgments}

profs grants

If these people aren't authors then we need to ack them
\begin{itemize}
    \item Owen 
    \item Eric
    \item People from the original PPLT project
\end{itemize}

\ally{don't forget GRFP} 



\balance

\bibliographystyle{plain}
\bibliography{misc/bibliography.bib}

\appendix
\onecolumn
\section{Additional Figures}\label{app:morefigs}


% that's all folks
\end{document}
